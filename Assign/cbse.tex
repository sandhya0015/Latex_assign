
\documentclass[12pt,-letter paper]{article}
\usepackage{siunitx}
\usepackage{setspace}
\usepackage{gensymb}
\usepackage{xcolor}
\usepackage{caption}
%\usepackage{subcaption}
\doublespacing
\singlespacing
\usepackage[none]{hyphenat}
\usepackage{amssymb}
\usepackage{relsize}
\usepackage[cmex10]{amsmath}
\usepackage{mathtools}
\usepackage{amsmath}
\usepackage{commath}
\usepackage{amsthm}
\interdisplaylinepenalty=2500
%\savesymbol{iint}
\usepackage{txfonts}
%\restoresymbol{TXF}{iint}
\usepackage{wasysym}
\usepackage{amsthm}
\usepackage{mathrsfs}
\usepackage{txfonts}
\let\vec\mathbf{}
\usepackage{stfloats}
\usepackage{float}
\usepackage{cite}
\usepackage{cases}
\usepackage{subfig}
%\usepackage{xtab}
\usepackage{longtable}
\usepackage{multirow}
%\usepackage{algorithm}
\usepackage{amssymb}
%\usepackage{algpseudocode}
\usepackage{enumitem}
\usepackage{mathtools}
%\usepackage{eenrc}
%\usepackage[framemethod=tikz]{mdframed}
\usepackage{listings}
%\usepackage{listings}
\usepackage[latin1]{inputenc}
%%\usepackage{color}{   
%%\usepackage{lscape}
\usepackage{textcomp}
\usepackage{titling}
\usepackage{hyperref}
%\usepackage{fulbigskip}   
\usepackage{tikz}
\usepackage{graphicx}
\lstset{
  frame=single,
  breaklines=true
}
\let\vec\mathbf{}
\usepackage{enumitem}
\usepackage{graphicx}
\usepackage{siunitx}
\let\vec\mathbf{}
\usepackage{enumitem}
\usepackage{graphicx}
\usepackage{enumitem}
\usepackage{tfrupee}
\usepackage{amsmath}
\usepackage{amssymb}
\usepackage{mwe} % for blindtext and example-image-a in example
\usepackage{wrapfig}
\graphicspath{{figs/}}
\providecommand{\mydet}[1]{\ensuremath{\begin{vmatrix}#1\end{vmatrix}}}
\providecommand{\myvec}[1]{\ensuremath{\begin{bmatrix}#1\end{bmatrix}}}
\providecommand{\cbrak}[1]{\ensuremath{\left\{#1\right\}}}
\providecommand{\brak}[1]{\ensuremath{\left(#1\right)}}



\begin{document}

\begin{enumerate}
	\item    For what values of $ k, $ the system of linear equation
      \begin{enumerate}
      \item $ x+y+z+=2 $
      \item $ 2x+y-z=3 $
      \item $ 3x+2y+k=4 $ \\
       has a unique solution ?      
      \end{enumerate}
\item If $\overrightarrow{a}=$ $4\hat{i}-\hat{j}+\hat{k}$ and $\overrightarrow{b}=$ $2\hat{i}-2\hat{j}+\hat{k}$, them find a unit vector parallel to the vector $\overrightarrow{a}+\overrightarrow{b}.$
\item Find $\lambda$ and $\mu$ if \\
	$\brak{\hat{i}+3\hat{j}+9\hat{k}}\times\brak{3\hat{i}-\lambda\hat{j}+\mu\hat{k}}=\overrightarrow{0}$.
\item Write the sum of intercepts cut off by the plane $\overrightarrow{r}$. $\brak{2\hat{i}+\hat{j}-\hat{k}}-5=0$ on the three axis.
\item If $A$ = $\myvec{cos \alpha & sin \alpha \\ -sin \alpha & cos \alpha}, find\ \ \alpha\ \ satisfying\ \ 0< \alpha < \frac{\pi}{2}$\ \
. when\  \ $A+A^{T}=\sqrt{2}I_{2}$; \\ where $A^{T}$ is traspose of A.
\item If $A$ is a $3\times3$ matrix and $\mydet{3A}=k\mydet{A}$, then write the value of $k.$ 	
\item Find:$\int$\brak{x+3}$\sqrt{3-4x-x^2}$ dx.	
\item Evaluate: $\int_{-2}^{2}$ $\frac{x^2}{1+5x}$ dx.
\item Find the equation of tangents to the curve $y=x^3+2x-4$, which are perpendicular to line $x+14y+3=0 $
\item \[ If f(x)=
	\begin{cases}
		\frac{\sin(a+1)x + 2\sin x}{x},& x<0 \\
		        2, &x = 0 \\
		\frac{\sqrt{1+bx-1}}{x}, & x>0
	\end{cases}
		\]
\text{is continuous at x=0, then find the values of a and b.}
\item Solve for $x:tan^{-1}$ $\brak{x-1}+tan^{-1}\brak{x+1}=tan^{-1}$ 3x.
\item prove that $tan^{-1}$ $[\frac{6x-8x^3}{1-12x^2}]$
$-tan^{-1}$ $[\frac{4x}{1-4x^2}]$=$tan^{-1}$2x; $\mydet{2x}$ $<$ $\frac{1}{\sqrt{3}}.$
\item If x $cos(a+y)$=$cosy$ then prove that $\frac{dy}{dx}$ = $\frac{cos^2\brak{a+y}}{sina}$. \\
Hence show that sina $\frac{d^2y}{dx^2}$+sin2\brak{a+y}$\frac{dy}{dx}=0$.
\item Find $\frac{dy}{dx}$ if $y=sin^{-1}[\frac{6x-4\sqrt{1-4x^2}}{5}]$
\item $A$ bag $X$ contains $4$ white balls and $2$ black balls, while another bag Y contains $3$ white balls and
$3$ black balls. Two balls are drawn \brak{without replacement} at random from one of the bags and were found to be one white and one black. Find the probability that the balls were drawn from bag Y.
\item $A$ and $B$ throw a pair of dice alternately, till one of them gets a total of $10$ and wins the  game. 
Find their respective probabilities of winning, if A starts first.
\item Find the coordinates of the foot of perpendicular
drawn from the point $A\brak{-1,8,4}$ to the line joining the points $B\brak{0,-1,3}$ and $C\brak{2,-3,-1}$. Hence find the image of the $A$ in the line $BC$.
\item Show that the four points $A\brak{4,5,1}$, $B\brak{0,-1, 1}$, $C\brak{3,9,4}$ and $D\brak{-4,4,4}$ are coplanar.
\item A typist charges \rupee $145$ for typing $10$ English and $3$ Hind pages,while charges for typing $3$ english and $10$ Hindi pages are \rupee $180$. Using matrices, find the charges of typing one English and one Hindi page separately. However typist charged only \rupee $2$ per page from a poor student Shyam for $5$ Hindi pages. How much less was charged from this poor boy ? Which values are reflected in this problem ? 
\item Find the particular solution of the differential equation \\ $2y$ $e^{\frac{x}{y}}$ dx+\brak{y-2x} $e^{\frac{x}{y}} dy={0}$ \\ given that $x=0$ when $y=1$.
\item Find the particular solution of differential equation : $\frac{dy}{dx}=-\frac{x+ycosx}{1+sinx}$ \\ given that y=1 when x=0.
\item Find : $\int\frac{\brak{2x-5}e^{2x}}{\brak{2x-3}^3}dx$
\item Find : $\int\frac{x^2+x+1}{\brak{x^2+1}\brak{x+2}}dx$
\item Prove that $y=\frac{4sin\theta}{2+cos\theta}-\theta$ is an increasing function of $\theta$ on $[0,\frac{\pi}{2}]$.
\item Show that semi-vertical angle of a cone of maximum volume and given slant height is $\cos^{-1}\brak{\frac{1}{\sqrt{3}}}$
\item Using properties of determinants, prove that \\
$\myvec{\brak{x+y}^2 & zx & zy \\zx & \brak{z+y}^2 & xy \\zy & xy & \brak{z+x}^2} = 2xyz\brak{x+y+z}^3$
\item If $A$ = $\myvec{1&0&2&\\0&2&1\\2&0&3&}$ and $A^3-6A^2+7A+kI_3 = O find k$
\item Using the method of integration, find the area of the triangular region whose vertices are $\brak{2, -2}$,$\brak{4, 3}$ and $\brak{1, 2}$.
\item Let $A=R \times R$ and * be a binary operation on A defined by \\ $\brak{a, b}*\brak{c, d}=\brak{a+c, b+d}$ \\ Show that * is commutative and associative. \\ Find the identity element for * on $A$ Also find the inverse of every element $\brak{a, b}\epsilon A$.
\item Three numbers are selected at random (without replacement) from first six positive integers. Let X donete the largest of the three numbers obtained. Find the probability di.stribution of X. Also, find the mean and variane of the distribution.
\item A retired person wants to invest an amount of \rupee $50,000$. His broker recommends investing in two type of bonds $'A'$ and $'B'$ yielding $10\%$ and $90\%$ return respectively on the invested amount. He decides to invest at least \rupee $20,000$ in bond $'A'$ and at least \rupee $10,000$ in bond $'B'$. He also wants to invest at leasst as much in bond $'A'$ as in bond $'B'$. Solve this linear  programming problem graphically to maximise his returns. 
\item Find the equation of the plane which contains the line of intersection of the planes. \\ $\overrightarrow{r}$.$\brak{\hat{i}-2\hat{j}-3\hat{k}}-4=0$ and \\ $\overrightarrow{r}$.$\brak{-2\hat{i}+\hat{j}+\hat{k}}+5=0$ \\ and whose interrcept on x-axis is equal to that of on y-axis.	

\end{enumerate}
\end{document}
